\documentclass[12pt, oneside, article, a4paper]{memoir}


\usepackage{bm}
\usepackage{physics}
\usepackage{geometry}
\usepackage{amsmath, amsthm, amssymb}
\usepackage{microtype}
\usepackage{mathtools}
\usepackage{tikz}
\usepackage{tikz-cd}

\usepackage{varioref}
\usepackage{hyperref}
\usepackage[capitalize, noabbrev]{cleveref}

\newcommand{\mat}[1]{\bm{#1}}
\linespread{1.25}

\title{ \textsc{Mandatory Assignment 4} \\
\textsc{MATINF4130}}
\author{Ivar Stangeby}

\begin{document}
\maketitle

\chapter{Introduction}

In this assignment we take a look at the PageRank algorithm, that helped
establish \textsc{Google} as a powerful search engine.  The algorithm
assigns to each web page its ``popularity'' in a way that mimics how a
human would define a popular web page. Before discussing the algorithm
itself, we start with a mathematical intermezzo.

\chapter{Mathematical framework}

We first establish some notation. Let \( \mathcal{S} \) denote the unit
simplex
\begin{equation}
    \mathcal{S} \coloneqq \left\{ \mat{x} \in \mathbb{R}^n \mid x_i \geq
    0 \text{ for  } i = 1, \ldots, n \text{ and } \sum_{i=1}^n x_i =
1\right\}.
\end{equation}
For later, we note that this is a closed and bounded set, which in \(
\mathbb{R}^n\) is equivalent to compact. We let \( \mat{A} \) be a real
\( n \times n \) matrix with non-negative elements \( a_{ij} \geq 0 \),
whose columns sum to one, and refer to this as a \emph{stochastic
matrix}.  The image of \( \mathcal{S} \) under \( \mat{A} \) is denoted
\begin{equation}
    \mat{A}(\mathcal{S}) \coloneqq  \left\{ \mat{Ax} \mid \mat{x} \in
    \mathcal{S} \right\}.
\end{equation}

\begin{enumerate}[a)]
    \item If \( \mat{y} \in \mat{A}(\mathcal{S}) \) then \( \mat{y} = \mat{Ax}
        \) for some \( \mat{x} \). Note that since both \( x_i \) and \( a_{ij}
        \) are non-negative, we must have \( y_i \) non-negative for \( i = 1,
        \ldots, n \). The sum
        \begin{equation}
            \sum_{i=1}^n y_i = \sum_{j=1}^n x_{j} \Big(\sum^{n}_{i=1}
            a_{ij}\Big) = \sum_{j=1}^n x_j = 1
        \end{equation}
        tells us that \( \mat{y} \in \mathcal{S} \) and consequently \(
        \mat{A}(\mathcal{S}) \subseteq \mathcal{S} \).

    \item Considering \( \mat{A} \colon \mathcal{S} \to \mathcal{S} \) as a
        linear operator, it suffices to show that it is bounded to show
        continuity. We have that
        \begin{equation}
            \norm{\mat{A}}_1 = \max_{\norm{x} = 1}\norm{\mat{Ax}}_1 = 1
        \end{equation}
        so \( \mat{A} \) is bounded, and therefore also continuous in the \(
        \norm{\cdot}_1 \) norm.

    \item Assume that \( (\lambda, \mat{v}) \) is an eigenpair for \( \mat{A}
        \). Since \( \mat{Av} = \lambda \mat{v} \in \mathcal{S} \), we must
        have \( |\lambda| \leq 1 \). Since \( \mathcal{S} \) is closed and
        bounded it is compact, and since \( \mathcal{S} \) is continuous, it
        follows by Brouwer's fixed-point theorem that there exists a  \(
        \mat{w} \) such that
        \begin{equation}
            \mat{Aw} = \mat{w}.
        \end{equation}
        Consequently, \( (1, \mat{w}) \) is a right eigenpair for \( \mat{A}
        \).

        From now on, we assume that the matrix entries \( a_{ij}  \) are all
        strictly positive. Denote by \( \mathcal{S}^\star \) the
        \emph{interior} of \( \mathcal{S} \):
        \begin{equation}
            \mathcal{S}^\star \coloneqq \left\{ \mat{x} \in \mathcal{S}  \mid
            x_i > 0 \text{ for  } i = 1, \ldots, n\right\}.
        \end{equation}

    \item Let \( \mat{x} \in \mathcal{S} \) and set \( \mat{y} = \mat{Ax} \).
        Since at least one of the elements \( x_i \) are non-negative, and all
        \( a_{ij} \) are strictly positive, we have \( y_i > 0\) for all \( i =
        1, \ldots, n \). This means that \(\mat{y} \in \mathcal{S}^\star \), so \(
        \mat{A} \) maps \( \mathcal{S} \) to its interior.

    \item\label{it:contract}Let \( \mat{x}, \mat{y} \in \mat{S} \) be two distinct vectors. Since
        the components of \( \mat{x} \) and the components of \( \mat{y} \) sum
        to one, we have that the components of \( \mat{z} \coloneqq \mat{x} -
        \mat{y} \) sum to zero. This means that since \( \mat{x} \) and \(
        \mat{y} \) are different, \( \mat{z} \) is non-zero, hence \(z_j < 0 \)
        for at least one \( j \). We will need this fact to achieve a strict
        inequality. We have that
        \begin{align}
            \begin{split}
            \norm{\mat{Ax} - \mat{Ay}}_1 = \norm{\mat{Az}}_1 &= \sum_{i = 1}^n |\sum_{j=1}^n a_{ij} z_j| \\
                                                             &< \sum_{j = 1}^n \sum_{i=1}^n a_{ij}|z_j| \\
                                                             &= \sum_{j=1}^n |z_j| = \norm{\mat{z}}_1 = \norm{\mat{x} - \mat{y}}_1
    \end{split}
        \end{align}
        Consequently, \( \mat{A} \colon \mathcal{S} \to \mathcal{S} \) is a
        \emph{contraction} in the \( \norm{\cdot}_1 \) norm.  Assume that \(
        \mat{w}_1 \neq \mat{w}_2 \) are two distinct eigenvectors with
        eigenvalue one. Then
        \begin{equation}
            \norm{\mat{w}_1 - \mat{w}_2}_1 = \norm{\mat{Aw}_1 - \mat{Aw}_2}_1 < \norm{\mat{w}_1 - \mat{w}_2}_1,
        \end{equation}
        which is a contradiction. We can therefore conclude that the geometric
        multiplicity \( g(\lambda) \) of the eigenvalue \( \lambda = 1 \) is
        one.
\end{enumerate}

\chapter{The PageRank Algorithm}

\section{Motivation}

Any web page of has a set of \emph{forward links} which are links \emph{to}
other pages, and a set of \emph{backlinks} which are links \emph{from} other
pages. Intuitively, pages with a large amount of backlinks should be deemed as
more popular than those with few backlinks. There is however one flaw with this
interpretation, and that is if the set of backlinks all come from obscure,
rarely visited websites, the real popularity of the page is certainly lower.
However, if a page has a few backlinks, but these come from large popular
websites, they should be qualify as more important. The PageRank algorithm
attempts to formalize this.

\section{Notation}

In the following, we will assume we have \( n \) websites, and that the
\emph{rank} or \emph{popularity} of page \( j \) is denoted \( p_j \).  We
store the page ranks in a popularity vector \( \mat{p} \coloneqq [p_1, \ldots,
p_n]^T \) which we assume to be normalized since we are only interested in the
relative popularity. The total number of forward links from page \( j \) is
denoted \( \ell_j \) and the set of pages with forward links to page \( i \) is
denoted
\begin{equation}
B_i \coloneqq \left\{j \mid \text{ there is a link from page } j \text{ to page
} i \right\}.
\end{equation}
We let \( \mat{B} \) be the matrix with entries \(b_{ij} \) defined by
\begin{equation}
    b_{ij} \coloneqq \begin{cases}
        \frac{1}{\ell_j} & \text{if there is a link from page \( j \) to page \( i \)}, \\
        0 & \text{otherwise}
    \end{cases}
\end{equation}

\section{The Algorithm}

The idea is that the popularity of page \( i \) satisfies the following
relation:
\begin{equation}
    p_i = \sum_{j \in B_i} \frac{1}{\ell_j} p_j.
\end{equation}
That is, every page distributes its popularity among its forward links. We are
interested in finding the popularity vector \( \mat{p} \). Notice that this
relation is equivalent to the following matrix equation being satisfied
\begin{equation}
    \mat{B}\mat{p} = \mat{p},
\end{equation}
hence we are looking for an eigenvector \( \mat{p} \) for \( \mat{B} \) with
eigenvalue one.

\begin{enumerate}[a)]

    \item An initial question is, when does such an eigenvector exist? Based on
        the work we did in earlier, we know that such an element exist if \(
        \mat{B} \) turns out to be stochastic. The matrix \( \mat{B} \) is
        stochastic precicely when each page has atleast one outgoing link, as
        this ensures that none of the columns of \( \mat{B} \) are zero, and
        then each column sum to one by construction. As an example, the
        set of pages shown here
        \begin{equation}
            \begin{tikzcd}
        1 \arrow[rr] \arrow[rd] &  & 2 \arrow[ld] \arrow[dd] \\
         & 3 \arrow[ld] \arrow[rd] &  \\
        4 \arrow[uu] \arrow[rr, bend right] &  & 5 \arrow[uu, bend right] \arrow[ll]
            \end{tikzcd}
        \end{equation}
        is associated to a stochastic matrix \( \mat{B} \) given by
        \begin{equation}
            \mat{B} = \frac{1}{2} \begin{bmatrix}
                0 & 0 & 0 & 1 & 0 \\
                1 & 0 & 0 & 0 & 1 \\
                1 & 1 & 0 & 0 & 0 \\
                0 & 0 & 1 & 0 & 1 \\
                0 & 1 & 1 & 1 & 0
            \end{bmatrix}
        \end{equation}
        and solving the equation \( \mat{Bp} = \mat{p} \) yields a popularity vector of
        \begin{equation}
            \mat{p} \approx
        \end{equation}

    \item We want to formulate an iteration scheme for the approximate solution
        to the equation \( \mat{Bp} = \mat{p} \). We may start with an
        arbitrary vector \( \mat{p}^0 \in \mathcal{S} \), and define \(
        \mat{p}^{k+1}  \coloneqq \mat{Bp}^k \) for \( k = 0, 1, 2, \ldots \),
        however we are not guaranteed that \( \mat{B} \) is stochastic.
        We therefore introduce the modified matrix \( \mat{X} \) defined by
        \begin{equation}
            \mat{X} \coloneqq \mat{B} + \frac{1}{n} \mat{C}
        \end{equation}
        where \( \mat{C} \) is an averaging matrix that fills out any zero
        columns of \( \mat{B} \) so that the columns sum to one. That is,
        \begin{equation}
            c_{ij} \coloneqq \begin{cases}
                1, & \mat{B}_{:, j} = \mat{0},\\
                0 & \text{otherwise}.
            \end{cases}
        \end{equation}
        This ensures that the matrix \( \mat{X} \) is stochastic. In order to
        make sure that the iteration converges, we introduce another modified
        matrix \( \mat{A} \)
        \begin{equation}
            \mat{A} \coloneqq \alpha \mat{X} + (1 - \alpha) \frac{1}{n} \mat{1},
        \end{equation}
        where \( \alpha \in (0, 1)\) and \( \mat{1} \) is an \( n\times n\)
        matrix filled with ones. This modification ensures that \( \mat{A} \)
        is a contraction (cf. exercise \vref{it:contract}).  The sequence \(
        \left\{ \mat{p}^k \right\}_{k=1}^\infty \) with \( \mat{p}^0 \in
        \mathcal{S} \) and
        \begin{equation}
            \mat{p}^{k+1} = \mat{Ap}^k
        \end{equation}
        therefore converges to a unique \( \mat{p} \) that solves \( \mat{Bp} =
        \mat{p} \).  Writing out the iteration we have
        \begin{equation}
            \mat{p}^{k+1} = \alpha \mat{Bp}^k  + \frac{\alpha}{n} \mat{Cp}^k +
            \frac{1 - \alpha}{n} \mat{1p}^k
        \end{equation}
    
        \item
        The matrix products \( \mat{Cp}^k \) and \( \mat{1p}^k \) are large
        linear systems, and it is worthwile to spend some time considering
        them.  Since \( \mat{B} \) is known, we know exactly which columns of \(
        \mat{B} \) are zero. Assume there are \( k \) zero-columns in \(
        \mat{B} \) with indices of these columns as \( i_1, \ldots, i_k \). We
        may then set
        \begin{equation}
            \mat{Cp}^k(j) = p_{i_1} + \ldots + p_{i_k} 
        \end{equation}
        for \( j = 1, \ldots, n \).  Similarly, we may directly compute the
        elements of \( \mat{1p}^k \) as
        \begin{equation}
            \mat{1p}^k(j) = p_1 + \ldots + p_n 
        \end{equation}
        for all \( j \).

        \item
            We test the PageRank algorithm on the supplied set of data. For
            each iteration we compute the residual error, to see whether the
            method converges or not. \cref{fig:convergence} shows the residual
            error plotted against the iterations.
            \begin{figure}[htbp]
                \centering
                \includegraphics[width=0.8\linewidth]{convergence.pdf}
                \caption{The residual error \( E_{k+1} \coloneqq
                \norm{\mat{p}^{k+1} - \mat{p}^k} \). We see that the method has
            stabilized after approximately 20 iterations.}
                \label{fig:convergence}
            \end{figure}
            We used the PageRank algorithm to determine the ten most popular
            webpages, and these are displayed in \cref{tbl:most_popular}.
            \begin{table}[htbp]
                \centering
                \caption{The ten most popular webpages computed by the PageRank
                algorithm, with 50 iterations.}
                \label{tbl:most_popular}
                \begin{tabular}{lll}
                    \toprule
                    Position & ID & Popularity[\%] \\
                    \midrule
                    1 & 6982 &	 0.103  \\
                    2 & 2130 &	 0.0807 \\
                    3 & 1752 &	 0.0756 \\
                    4 & 8096 &	 0.0706 \\
                    5 & 8565 &	 0.0702 \\
                    6 & 3735 &	 0.0635 \\
                    7 & 1578 &	 0.0585 \\
                    8 & 8268 &	 0.0579 \\
                    9 & 6917 &	 0.0572 \\
                    10 & 6553 &	 0.0569 \\
                    \bottomrule
                \end{tabular}
            \end{table}
        \end{enumerate}
\end{document}


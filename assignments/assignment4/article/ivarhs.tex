\documentclass[12pt, oneside, article, a4paper]{memoir}


\usepackage{bm}
\usepackage{physics}
\usepackage{geometry}
\usepackage{amsmath, amsthm, amssymb}
\usepackage{microtype}
\usepackage{mathtools}

\newcommand{\mat}[1]{\bm{#1}}
\linespread{1.25}

\title{ \textsc{Mandatory Assignment 4} \\
\textsc{MATINF4130}}
\author{Ivar Stangeby}

\begin{document}
\maketitle

\chapter*{Introduction}

In this assignment we take a look at the PageRank algorithm, that helped
establish \textsc{Google} as a powerful search engine.  The algorithm
assigns to each web page its ``popularity'' in a way that mimics how a
human would define a popular web page. Before discussing the algorithm
itself, we start with a mathematical intermezzo.

\paragraph*{Exercise 1}

We first establish some notation. Let \( \mathcal{S} \) denote the unit
simplex 
\begin{equation}
    \mathcal{S} \coloneqq \left\{ \mat{x} \in \mathbb{R}^n \mid x_i \geq
    0 \text{ for  } i = 1, \ldots, n \text{ and } \sum_{i=1}^n x_i =
1\right\}.
\end{equation}
For later, we note that this is a closed and bounded set, which in \(
\mathbb{R}^n\) is equivalent to compact. We let \( \mat{A} \) be a real
\( n \times n \) matrix with non-negative elements \( a_{ij} \geq 0 \),
whose columns sum to one, and refer to this as a \emph{stochastic
matrix}.  The image of \( \mathcal{S} \) under \( \mat{A} \) is denoted
\begin{equation}
    \mat{A}(\mathcal{S}) \coloneqq  \left\{ \mat{Ax} \mid \mat{x} \in
    \mathcal{S} \right\}.
\end{equation}

\begin{enumerate}[a)]
    \item If \( \mat{y} \in \mat{A}(\mathcal{S}) \) then \( \mat{y} = \mat{Ax}
        \) for some \( \mat{x} \). Note that since both \( x_i \) and \( a_{ij}
        \) are non-negative, we must have \( y_i \) non-negative for \( i = 1,
        \ldots, n \). The sum
        \begin{equation}
            \sum_{i=1}^n y_i = \sum_{j=1}^n x_{j} \Big(\sum^{n}_{i=1}
            a_{ij}\Big) = \sum_{j=1}^n x_j = 1
        \end{equation}
        tells us that \( \mat{y} \in \mathcal{S} \) and consequently \(
        \mat{A}(\mathcal{S}) \subseteq \mathcal{S} \).

    \item Considering \( \mat{A} \colon \mathcal{S} \to \mathcal{S} \) as a
        linear operator, it suffices to show that it is bounded to show
        continuity. We have that
        \begin{equation}
            \norm{\mat{A}}_1 = \max_{\norm{x} = 1}\norm{\mat{Ax}}_1 = 1
        \end{equation}
        so \( \mat{A} \) is bounded, and therefore also continuous in the \(
        \norm{\cdot}_1 \) norm.
    
    \item Assume that \( (\lambda, \mat{v}) \) is an eigenpair for \( \mat{A}
        \). Since \( \mat{Av} = \lambda \mat{v} \in \mathcal{S} \), we must
        have \( |\lambda| \leq 1 \). Since \( \mathcal{S} \) is closed and
        bounded it is compact, and since \( \mathcal{S} \) is continuous, it
        follows by Brouwer's fixed-point theorem that there exists a  \(
        \mat{w} \) such that
        \begin{equation}
            \mat{Aw} = \mat{w}.
        \end{equation}
        Consequently, \( (1, \mat{w}) \) is a right eigenpair for \( \mat{A}
        \).

        From now on, we assume that the matrix entries \( a_{ij}  \) are all
        strictly positive. Denote by \( \mathcal{S}^\star \) the
        \emph{interior} of \( \mathcal{S} \):
        \begin{equation}
            \mathcal{S}^\star \coloneqq \left\{ \mat{x} \in \mathcal{S}  \mid
            x_i > 0 \text{ for  } i = 1, \ldots, n\right\}.
        \end{equation}
        
    \item Let \( \mat{x} \in \mathcal{S} \) and set \( \mat{y} = \mat{Ax} \).
        Since at least one of the elements \( x_i \) are non-negative, and all
        \( a_{ij} \) are strictly positive, we have \( y_i > 0\) for all \( i =
        1, \ldots, n \). This means that \(\mat{y} \in \mathcal{S}^\star \), so \(
        \mat{A} \) maps \( \mathcal{S} \) to its interior.
    
    \item Let \( \mat{x}, \mat{y} \in \mat{S} \) be two distinct vectors. Since
        the components of \( \mat{x} \) and the components of \( \mat{y} \) sum
        to one, we have that the components of \( \mat{z} \coloneqq \mat{x} -
        \mat{y} \) sum to zero. This means that since \( \mat{x} \) and \(
        \mat{y} \) are different, \( \mat{z} \) is non-zero, hence \(z_j < 0 \)
        for at least one \( j \). We will need this fact to achieve a strict
        inequality. We have that
        \begin{align}
            \begin{split}
            \norm{\mat{Ax} - \mat{Ay}}_1 = \norm{\mat{Az}}_1 &= \sum_{i = 1}^n |\sum_{j=1}^n a_{ij} z_j| \\
                                                             &< \sum_{j = 1}^n \sum_{i=1}^n a_{ij}|z_j| \\ 
                                                             &= \sum_{j=1}^n |z_j| = \norm{\mat{z}}_1 = \norm{\mat{x} - \mat{y}}_1
    \end{split}
        \end{align}
        Consequently, \( \mat{A} \colon \mathcal{S} \to \mathcal{S} \) is a
        \emph{contraction} in the \( \norm{\cdot}_1 \) norm.  Assume that \(
        \mat{w}_1 \neq \mat{w}_2 \) are two distinct eigenvectors with
        eigenvalue one. Then
        \begin{equation}
            \norm{\mat{w}_1 - \mat{w}_2}_1 = \norm{\mat{Aw}_1 - \mat{Aw}_2}_1 < \norm{\mat{w}_1 - \mat{w}_2}_1,
        \end{equation}
        which is a contradiction. We can therefore conclude that the geometric
        multiplicity \( g(\lambda) \) of the eigenvalue \( \lambda = 1 \) is
        one.
\end{enumerate}



\end{document}

\documentclass[12pt, oneside, a4paper, article]{memoir}

\usepackage{microtype}
\usepackage{geometry}
\linespread{1.25}
\usepackage{amsmath, amsthm, amssymb}
\usepackage{thmtools, mathtools, commath}
\usepackage{minted}
\usepackage{hyperref}
\usepackage[noabbrev, capitalize]{cleveref}


\newcommand{\mat}[1]{\boldsymbol{#1}}

\title{ \textsc{Mandatory Assignment 1 \\
MAT-INF4130}}
\author{Ivar Haugal{\o}kken Stangeby}
\begin{document}

    \maketitle 
       
    \chapter*{Problem 1} 

    In this problem we consider how to invert a lower triangular matrix. Recall
    that a lower triangular matrix has a lower triangular inverse. Assume that
    \( \mat{A} \in \mathbb{C}^{n\times n} \) is a lower triangular matrix with
    \( \mat{B} \) its lower triangular inverse. In terms of block matrices we
    have that
    \begin{equation}
        \mat{A}\mat{B} = \mat{A}[\mat{b}_1, \ldots, \mat{b}_n] = [\mat{e}_1,
        \ldots, \mat{e}_n] = \mat{I}.
    \end{equation}
    Considering equation \( \mat{A}\mat{b}_k = \mat{e}_k \) and looking at the
    \( k \)-th row in \( \mat{A} \) we see that
    \begin{equation}
        \sum_{i = 1}^n a_{ki}b_{ii} = 1.
    \end{equation}
    Since both \( \mat{A} \) and \( \mat{B} \) are lower triangular, this sum
    reduces to \( a_{kk}b_{kk} = 1 \), and consequently \( b_{kk} = 1 / a_{kk}
    \) for all \( k = 1, \ldots, n \).  In order to compute the rest of \(
    \mat{b}_k \), consider the following system of matrices
    \begin{equation}
        \label{eq:kthstep}
        \begin{bmatrix}
            \mat{A}_{[k]} & \mat{0} \\
            \mat{C} & \mat{\tilde{A}}
        \end{bmatrix}
        \begin{bmatrix}
            \mat{b}_k^1 \\
            \mat{b}_k^2
        \end{bmatrix}
        = 
        \mat{e}_k.
    \end{equation}
    Here \( \mat{A}_{[k]} \) is the \emph{principal leading submatrix} of \(
    \mat{A} \), while
    \begin{equation}
        \mat{b}_{k}^1 = [0, \ldots, 0, b_{kk}], \qquad \mat{b}_{k}^2 =
        [b_{k+1, k}, \ldots, b_{nk}].
    \end{equation}
    Expanding this system yields the equation
    \begin{equation}
        \mat{\tilde{A}} \mat{b}_{k}^2 = -\mat{C}\mat{b}_k^1, 
    \end{equation}
    where the right hand side reduces to \( -\mat{A}((k+1)\mathord{:}n,
    k)b_{kk} \) by the properties of \( \mat{e}_k \). We can therefore compute
    \( \mat{B} \) by computing the diagonal elements, then solve the linear
    system
    \begin{equation}
        \label{eq:linsystem}
        \mat{A}((k+1)\mathord{:}n,
        (k+1)\mathord{:}n)\mat{b}_k((k+1)\mathord{:}n) = -
        \mat{A}((k+1)\mathord{:}n, k)b_{kk}
    \end{equation}
    for \( k = 1, \ldots, n - 1\).
    
    \subsection{Numerical Considerations} 
    
    For numerical efficiency it is feasible to do these computations inplace.
    From \cref{eq:kthstep} we see that in the \(k\)-th step of the computation
    we do not use the top right part of the matrix. Furthermore, the diagonal
    elements are only used once and not in any subsequent computations, hence
    we may safely replace these values by others. We can therefore store the
    vectors \( \mat{b}_k^2 \) in place of \( \mat{A}(k, k \mathop{:} n) \).
    After the computations are done, \( \mat{B} \) may be extracted from \(
    \mat{A} \) by transposing and zeroing out the upper diagonals.

    \chapter*{Problem 2}
    
    We are interested in the number of operations for computing the inverse of
    a lower triangular matrix \( \mat{A} \in \mathbb{C}^{n\times n}\) following
    the algorithm described above.  For \( k = 1, \ldots, n - 1 \), we compute
    the diagonal element \( b_{kk} \) and solve the corresponding linear system
    as given in \cref{eq:linsystem}. Each such system can be solved using a
    solver relying on the fact that the matrix is lower triangular. Hence for
    \( i = 1 , \ldots, n - k \) there are \( i \) multiplications and \( i - 1
    \) additions. Letting \( N \) denote the total number of operations, we have
    \begin{align}
        N &= \sum_{k = 1}^{n-1} \sum_{i = 1}^{n-k} 2i - 1 = \sum_{k=1}^{n-1}(n-k)^2
        \intertext{and the leading term is given by}
        &\approx \int_{1}^{n-1} (n-k)^2 \dif{k} \approx \frac{1}{3}n^3.
    \end{align}
    The algorithm therefore has a time complexity of \( \mathcal{O}(n^3) \).
    
    \chapter*{Problem 3}
    
    \begin{listing}
        \begin{minted}{python}
from numpy import *

n = 8
A = matrix(random.random((n, n)))
A = triu(A)
U = A.copy()
for k in range(n-1, -1, -1):
  U[k, k] = 1 / U[k, k]
  for r in range(k-1, -1, -1):
     U[r, k] = -U[r, (r+1) : (k+1)] * U[(r+1) : (k+1), k] / U[r, r]
print(U*A)
        \end{minted}
        \caption{Computing the upper triangular inverse of an upper triangular
        matrix. Implemented in \textsc{Python}.}
        \label{lst:code}
    \end{listing}
    We are tasked with discussing the code snipped given in \cref{lst:code}.
    This is a \textsc{Python}-implementation of the algorithm discussed above for
    computing the inverse of a triangular matrix. In this case, an upper
    triangular one. The variables \texttt{r} and \texttt{k} represent the row
    and column index respectively. Note that the computation is performed
    in-place, and the inverse of \( \mat{A} \) is stored in memory allocated to
    storing \( \mat{A} \). The final product \texttt{U * A} is the identity
    matrix.
 
\end{document}
